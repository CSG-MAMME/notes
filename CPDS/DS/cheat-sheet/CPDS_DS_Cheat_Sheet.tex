\documentclass[a4paper, 10pt, twocolumn]{article}
\usepackage[utf8]{inputenc}
\usepackage[a4paper, left=0.75cm, right=0.75cm, top=0.5cm, bottom=0.75cm, headsep=0.25cm, includehead]{geometry}
\usepackage{url}
\usepackage{graphicx}
\usepackage{color}
\usepackage{amsmath}
\usepackage{fancyhdr}
\pagestyle{fancy}
\lhead{CPDS: Distributed Systems Module - Cheat Sheet}
\rhead{Carlos Segarra}
\pagenumbering{gobble}
\usepackage[linkcolor=blue]{hyperref}
\usepackage{enumitem}
\setlist[enumerate]{noitemsep, topsep=2pt}
\setlist[itemize]{noitemsep, topsep=2pt}

\begin{document}

    \textbf{Lin's Voting Algorithm}
    
    Decentralized algorithm with \textbf{N coordinators}.
    Using \textbf{voting}, gain quorum from the coordinators.
    \begin{enumerate}
        \item Process sends \textbf{Request} to each coordinator: including Lamport's clock and ID.
        \item Coordinator grants vote if it has not voted yet (otherwise queue request) and send \textbf{Response} w/ current vote.
        \item Requester analyzes votes in responses: if majority accesses section, if someone else majority waits. Else, either releases and backs off, or sends \textbf{yield} (release + request).
        \item On receiving a \textbf{yield} a coordinator removes its vote, queues request and sends response.
        \item When the process exists the critical section sends a \textbf{release} message.
    \end{enumerate}

    \textbf{Ricart \& Agrawala's Voting Algorithm}

    Multicast with logical clocks.
    \begin{enumerate}
        \item Process sends \textbf{Request} to all other processes.
        \item When requester receives \textbf{OK} from all processes, it can access the critical section.
        \item When a process receives a \textbf{Request}: if not accessing send \textbf{OK}, if accessing now queue, if wants to access decide basing on clock + PID.
    \end{enumerate}
    Variant: get permission to access CS by obtaining votes from a \textbf{subset} of the rest of the processes.
    All voting sets have the same size $K$, and each $p_i$ belongs to $M$ voting sets, and is ruled by just one (opt. at $K = M \simeq \sqrt{N}$).

    \textbf{Bully Election Algorithm}

    Synchronous system and reliable message delivery.
    \begin{enumerate}
        \item P sends an \textbf{Election} message to all the processes with higher IDs and waits for \textbf{OK} messages.
        \item If a process receives an \textbf{Election} message, returns \textbf{OK} and starts another election.
        \item If P receives an \textbf{OK} it drops out the election and awaits a \textbf{Coordinator} message.
        \item If P does not receive any \textbf{OK} before the timeout (synchronous system), it wins and sens a \textbf{Coordinator} message.
    \end{enumerate}

    \textbf{Chang and Roberts' Ring Election Algorithm}

    Processes are organized by ID in a logical unidirectional ring (each process knows its successor), asynchronous system.
    \begin{enumerate}
        \item P sends an \textbf{Election} message (with its ID) to its successor, and becomes a participant.
        \item On receiving an \textbf{Election}, Q compares the ID in the message with its own: if greater Q forwards the message, if smaller Q replaces the ID with its own and becomes a participant (unless it already is, in which case it drops the message). If the ID is the same, Q wins and sends a \textbf{Coordinator} message.
        \item When Q receives a \textbf{Coordinator} message, it forwards it and becomes a non-participant.
    \end{enumerate}
    What happens when a process crashes? We make use of the \textbf{Enhanced Ring Algorithm}.
    \begin{enumerate}
        \item P sends an \textbf{Election} message to its closest alive successor.
        \item At each step, each process adds its ID to the list in the message.
        \item When the message gets back to the initiator (first PID) it elects the highest ID in the list and sends a \textbf{Coordinator} with this ID.
        \item When relaying, each process adds its ID to the list, if when finished the elected process ID is in the list, election is over.
    \end{enumerate}

    \textbf{Basic Reliable Multicast}

    Assume process do not fail and do not join/leave the group.
    \begin{enumerate}
        \item Sender $P$ assigns a sequence number to each outgoing message, and stores a copy until everyone has acked.
        \item Each process $Q$ records the number of the \textbf{last message it has delivered} coming from any other process $P$, $L_Q(P)$.
        \item When $Q$ receives a message from $P$: if $S_P = L_Q(P) + 1$ it delivers the message, increase counter, and ack $P$, if it is greater, keeps message in queue and requests RTX of missing elements, otherwise it discards the message.
    \end{enumerate}

    \textbf{Scalable Reliable Multicast}

    Use sequence numbers but \textbf{reduce} the number of feedback messages to the sender.
    Only missing messages are reported.
    Each process waits a random delay prior to sending a NACK, and if they receive one, they discard theirs.

    \textbf{Ordered Multicast}

    Due to latency, messages might arrive in different order at different nodes.
    Common ordering requirements: FIFO (preserve intra-process order), causal (happened-before relation), total (all process deliver in same order).
    In order to implement it:
    \begin{itemize}
        \item \textbf{FIFO:} Use \textbf{sequence numbers per sender} and hold-back queue for out-of-order messages.
        \item \textbf{Total:} either use a centralized sequencer to give a total ordering, or by agreement of the processes:
        \begin{enumerate}
            \item Sender multicasts message $m$.
            \item Each receiver $j$ replies with a proposed message number $P_j = \max(A_j, P_j) = \max(\text{ largest agreed }, \text{ own largest proposed })$.
            \item Sender selects the largest proposal $N$ and multicasts it.
            \item Receiver $j$ updates $A_j$ and reorders messages in the queue.
        \end{enumerate}
        \item \textbf{Causal:} delivered only if all causally preceedin messages have been delivered.
        \begin{enumerate}
            \item $P_i$ increases $VC_i[i]$ only when sending a message.
            \item If $P_j$ receives $m$ from $P_i$ it postpones delivery until: $ts(m)[i] = VC_j[i] + 1$ and $ts(m)[k] \leq VC_j[k] \forall k \neq i$.
            \item $P_j$ increases $VC_j[i]$ after delivering $m$.
        \end{enumerate}
    \end{itemize}

    \textbf{Dolev \& Strong's Consensus Algorithm}

    Crash-faulty processes and reliable communication in a synchronous system.
    Up to $f$ of the $N$ processes exhibit crash failures.
    Algorithm proceeds in $f+1$ rounds:
    \begin{enumerate}
        \item Each process proposes a value.
        \item From round $1$ to round $f+1$ each process:
        \begin{enumerate}
            \item Multicasts NEW values. Initially send its proposed value.
            \item Collects vales from other processes. (Round terminates when values from all processes are collected or by timeout).
        \end{enumerate}
        \item Each process decides against collected values.
    \end{enumerate}

    \textbf{OM(m) Consensus Algorithm}

    Byzantine-faulty processes and reliable communication in a synchronous system.
    At most $m$ traitors and at least $3m + 1$ generals.
    It requires $m+1$ rounds.
    \begin{enumerate}
        \item \textbf{OM(0)}: the commander sends his value to every lieutenant, each lieutenant accepts the value he receives as the order.
        \item \textbf{OM(m), $m > 0$}:
        \begin{enumerate}
            \item The commander sends his value to every lieutenant.
            \item Each lieutenant $i$ acts as a commander in $OM(m-1)$ and broadcasts the value he got from the commander to the $n-2$ remaining lieutenants.
            \item Each lieutenant $i$ accepts the value $\text{majority}(v_1, \dots, v_{n-1})$ where $i$ is the directly received value, and the rest are the broadcasted values.
        \end{enumerate}
    \end{enumerate}

\end{document}
