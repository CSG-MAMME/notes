\section{Distributed Algorithms}

\subsection{Time \& Global States}
\subsubsection*{Time}
\textbf{Logical Clocks}
Implemented to capture the happened-before relation. They satisfy:
\begin{enumerate}
    \item If a and b are two events in the same process, a->b => C(a) < C(b)
    \item If a sends a message to b => C(a) < C(b)
\end{enumerate}

\textbf{Lamport's logical clocks:}
\begin{itemize}
    \item Each process $P_i$ has a counter $C_i$
    \item $C_i$ is updated using the following rules:
    \begin{enumerate}
        \item When an event happens at $P_i$ increment $C_i$ by one.
        \item When $P_i$ sends a message, set $ts(m) = C_i$
        \item When $P_i$ receives a message, set $C_i = \max(C_i, ts(m))$ and then increase $C_i$ by one. 
    \end{enumerate}
\end{itemize}
Lamport's clocks do not guarantee that if $C(a) < C(b) \Rightarrow a -> b$.

\textbf{Vector clocks:}
\begin{itemize}
    \item Each process $P_i$ has an array $VC_i [ 1, \dots, n ]$
    \item It is updated as follows:
    \begin{enumerate}
        \item When $P_i$ sends a message $m$, it adds 1 to $VC_i[i]$ and sends $m$ with $ts(m) = VC_i$.
        \item When $P_j$ receives a message from $P_i$, it updates each $VC_j[k]$ to $\max(VC_j[k], ts(m)[k])$ and increments $VC_j[j]$ by one.
    \end{enumerate}
\end{itemize}

\subsubsection*{Global States}
A global state of the system is necessary for:
\begin{itemize}
    \item Failure Recovery
    \item Detection of Properties: deadlocks, termination
    \item Debugging
\end{itemize}
We define some concepts:
\begin{enumerate}
    \item The \textbf{history} of a process $p$ is the sequence of events ocurred at that process: $h(p) = <p_0, p_1, \dots>$ (either internal or message sending).
    \item The \textbf{state} $i$ of process $p$ is p's history until event $i$: $s_i(p) = <p_0, \dots, p_i>$.
    \item The \textbf{global history} is the union of all the individual histories.
    \item A \textbf{cut} is the global history up to a specific event in each process history.
    \item A cut is \textbf{consistent} if it contains all the \textit{happened-before} events. A consitent cut corresponds to a \textbf{consistent global state}.
    \item A \textbf{run} is a total ordering of all events in a global history consistent with each local history.
    \item A \textbf{linearization} or \textbf{consistent run} is a run consistent with the \textit{happened-before} relation.
    \item We say state $S'$ is reachable from $S$ if there is a linearization such that $S$ preceeds $S'$.
\end{enumerate}

\textbf{Distributed Snapshot}

\textbf{Global Predicates}

Consistent global states form a lattice with reachability relation between sets.
A \textbf{global state predicate}, $\varphi$ is a property that is either true or false for a global state.
\begin{itemize}
    \item A predicate is \textbf{stable} if once it becomes true, it remains true for all reachable states.
    \item A predicate is \textbf{non-stable} if it can become true and then false.
    \item A predicate $\varphi$ possibly happened: if it is true for any of the consistent states in the lattice.
    \item A predicate $\varphi$ definately happened: if all paths from origin to end contain a consistent global state for which the predicate is true.
\end{itemize}

\subsection{Coordination and Agreement}

\subsubsection*{Mutual Exclusion}

\subsubsection*{Election Algorithms}

\subsubsection*{Multicast Communications}

\subsubsection*{Consensus}
