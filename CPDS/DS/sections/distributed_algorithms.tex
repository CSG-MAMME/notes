\section{Distributed Algorithms}

\subsection{Time \& Global States}
\subsubsection*{Time}

\textbf{Physical Clocks}

Physical clocks allow to synchronize nodes \textbf{within a given bound.} Synchronize at least every \textbf{\textcolor{red}{$R < \frac{\delta}{2\rho}$ to limit skew between two clocks to less than $\delta$}}.
Where:
\begin{itemize}
    \item $R$: resyncrhonization interval.
    \item $\rho$: maximum clock drift rate.
    \item $\delta:$ maximum allowed clock skew.
\end{itemize}


\textbf{Logical Clocks}

Implemented to capture the happened-before relation. They satisfy:
\begin{enumerate}
    \item If $a$ and $b$ are two events in the same process, $a \rightarrow b \Rightarrow C(a) < C(b)$
    \item If $a$ sends a message to $b => C(a) < C(b)$
\end{enumerate}

\textbf{Lamport's logical clocks:}

\begin{itemize}
    \item Each process $P_i$ has a counter $C_i$
    \item $C_i$ is updated using the following rules:
    \begin{enumerate}
        \item When an event happens at $P_i$ increment $C_i$ by one.
        \item When $P_i$ sends a message, set $ts(m) = C_i$
        \item When $P_i$ receives a message, set $C_i = \max(C_i, ts(m))$ and then increase $C_i$ by one. 
    \end{enumerate}
\end{itemize}
Lamport's clocks do not guarantee that if $C(a) < C(b) \Rightarrow a \rightarrow b$.

\textbf{Vector clocks:}

\begin{itemize}
    \item Each process $P_i$ has an array $VC_i [ 1, \dots, n ]$
    \item It is updated as follows:
    \begin{enumerate}
        \item When $P_i$ sends a message $m$, it adds 1 to $VC_i[i]$ and sends $m$ with $ts(m) = VC_i$.
        \item When $P_j$ receives a message from $P_i$, it updates each $VC_j[k]$ to $\max(VC_j[k], ts(m)[k])$ and increments $VC_j[j]$ by one.
    \end{enumerate}
\end{itemize}

\subsubsection*{Global States}
A global state of the system is necessary for:
\begin{itemize}
    \item Failure Recovery
    \item Detection of Properties: deadlocks, termination
    \item Debugging
\end{itemize}
We define some concepts:
\begin{enumerate}
    \item The \textbf{history} of a process $p$ is the sequence of events ocurred at that process: $h(p) = <p_0, p_1, \dots>$ (either internal or message sending).
    \item The \textbf{state} $i$ of process $p$ is p's history until event $i$: $s_i(p) = <p_0, \dots, p_i>$.
    \item The \textbf{global history} is the union of all the individual histories.
    \item A \textbf{cut} is the global history up to a specific event in each process history.
    \item A cut is \textbf{consistent} if it contains all the \textit{happened-before} events. A consitent cut corresponds to a \textbf{consistent global state}.
    \item A \textbf{run} is a total ordering of all events in a global history consistent with each local history.
    \item A \textbf{linearization} or \textbf{consistent run} is a run consistent with the \textit{happened-before} relation.
    \item We say state $S'$ is reachable from $S$ if there is a linearization such that $S$ preceeds $S'$.
\end{enumerate}

\textbf{Distributed Snapshot}

\textbf{Global Predicates}

Consistent global states form a lattice with reachability relation between sets.
A \textbf{global state predicate}, $\varphi$ is a property that is either true or false for a global state.
\begin{itemize}
    \item A predicate is \textbf{stable} if once it becomes true, it remains true for all reachable states.
    \item A predicate is \textbf{non-stable} if it can become true and then false.
    \item A predicate $\varphi$ possibly happened: if it is true for any of the consistent states in the lattice.
    \item A predicate $\varphi$ definately happened: if all paths from origin to end contain a consistent global state for which the predicate is true.
\end{itemize}

\subsection{Coordination and Agreement}

\subsubsection*{Mutual Exclusion}

\textbf{Problem:} A set of processes in a distributed system want exclusive access to some shared resource.
The desired properties are:
\begin{enumerate}
    \item \textbf{Safety:} at most one process may execute at a time.
    \item \textbf{Liveness:} requests to enter and exit eventually succeed.
    \item \textbf{Happened-before ordering}
\end{enumerate}

\begin{enumerate}
    \item Permission-Based Solutions:
    \begin{enumerate}
        \item \textbf{Centralized Algorithm:} a coordinator grants access to the shared resource, single point of failure.
        \item \textbf{Lin's Voting Algorithm:} decentralized algoritm with N coordinators.
        \item \textbf{Ricart \& Agrawala's Algorithm:} multicast with logical clocks, send request to all other processes and decide basing on logical time.
            Variant receiving votes from a subset of the processes ($M$ subsets of size $K$ being $\sqrt{N}$ the optimum).
    \end{enumerate}
    \item Token-Based Solutions:
    \begin{enumerate}
        \item Organize processes in a logical unidirectional ring.
        \item A \textbf{token} message circulates around the ring.
        \item Only the process holding the token can enter the critical section.
    \end{enumerate}
\end{enumerate}
\subsubsection*{Election Algorithms}
Election algorithms are techniques to pick a \textbf{unique} coordinator (leader).
Desired properties are:
\begin{enumerate}
    \item \textbf{Safety:} a participant is either non-decided or decided with the non-crashed process with the largest ID.
    \item \textbf{Liveness:} all processes eventually participate \& either decide a coordinator or crash.
\end{enumerate}
Any process can initiate an election (several ones may run concurrently).
Some algorithms:
\begin{itemize}
    \item \textbf{Bully Algorithm}
    \item \textbf{Chang and Robert's Ring Algorithm}
\end{itemize}
Some can tolerate failures, but none of them can deal with network partitions.

\subsubsection*{Multicast Communications}

It is an important service in distributed systems to disseminate data reliably to large number of users.
It is also used to implement several distributed algorithms.
Different types:
\begin{itemize}
    \item Multicast: send a message to a process group.
    \item Reliable Multicast: deliver messages to all or no process in the group.
    \item Ordered Multicast: deliver messages while fulfilling ordering requirements.
    \item Atomic Multicast: deliver messages in the same order to all processes and any process can fail.
        Solution for multicasing in open groups with \textbf{faulty} processes.
        Deliver messages only to \textbf{non-faulty} members.
        A membership service keeps all members updated on who the non-faulty members are (send \textbf{view messages} of group membership in total order).
        View changes when processes join/leave the group.
        Each message is associated with a group view (multicasts cannot pass across view changes).

\end{itemize}
\subsubsection*{Consensus}

General form of agreement: some processes must agree on a value in a finite number of steps in the presence of failures.
Some of the desired properties are:
\begin{itemize}
    \item \textbf{Termination:} Every non-faulty process must eventually decide.
    \item \textbf{Agreement:} The final decision of every non-faulty process must be identical.
    \item \textbf{Validity:} If all the non-faulty processes proposed the same value, then the final decision must be that value.
\end{itemize}

With \textbf{correct} processes and \textbf{reliable} communication, agreement is straightforward.
With \textbf{unreliable} communication, agreement \textbf{\textcolor{red}{cannot be guaranteed}}.
With \textbf{reliable communication}, crash-faulty processes, and synchronous system we use the \textbf{Dolev \& Strong's Algorithm}.
With \textbf{byzantine-faulty processes} and reliable communication in a synchronous system we face the \textbf{\textcolor{red}{Byzantine Generals Problem}}.
Byzantine processes may work together maliciously. We have interactive consistency requirements:
\begin{itemize}
    \item[IC1:] All loyal lieutenants obey the same order (agreement).
    \item[IC2:] If the commander is loyal, then every loyal lieutenant obeys the order he sends (integrity).
\end{itemize}
\textbf{Impossibility result:} with three processes, no solution can work with even one traitor.
\textbf{\textcolor{red}{No solution with fewer than $3m + 1$ generals can cope with $m$ traitors.}}

\textbf{Faulty} processes and \textbf{reliable} communication in an \textbf{asynchronous} system, no algorithm can guarantee to reach consensus.
