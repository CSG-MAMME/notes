\section{Concepts of Distributed Systems}

\subsection{Definition of a Distributed System}

A distributed system is a collection of \textbf{autonomous computing elements} that appears to its users as a \textbf{single coherent system}.

\subsection{Challenges of Distributed Systems}

\begin{enumerate}
    \item \textbf{Global Time:}
    \begin{itemize}
        \item Only low accuracy system clock.
        \item \textbf{Clock Skew:} two clocks, two times.
        \item \textbf{Clock Drift:} two clocks, two count rates.
    \end{itemize}
    \item \textbf{Asynchrony:} messages take (varaiable) time to be delivered.
        This results on two types of distributed systems.
    \begin{itemize}
        \item Synchronous: we use time and timeouts.
        \item Asynchronous: delays, execution time, and clock drift are \textbf{not bounded}.
    \end{itemize}
    \item \textbf{Transparency:} ability of presenting itself as a single computer system.
    \item \textbf{Fault Tolerance:} a failure is any deviation of the observed behaviour from the specified one.
    \begin{itemize}
        \item \textbf{Crash Failure:} process halts and remains halted.
        \item \textbf{Omission Failure:} sent message never arrives at the other end.
        \item \textbf{Timing Failure:} process response lies outside a time interval.
        \item \textbf{Response Failure:} process response is incorrect.
        \item \textbf{Byzantine Failure:} arbitrary failures (malicious).
    \end{itemize}
\end{enumerate}
