\section{MPI: Message Passing Interface}

\subsection{Environment}
The following commands are used to bootstrap and interact with the MPI runtime:
\begin{itemize}
    \item \texttt{int MPI\_Init(\&argc, \&argv)}: call once before any other MPI routine.
    \item \texttt{int MPI\_Comm\_size(MPI\_COMM\_WORLD, \&nproc)}: return the number of processes in the communicator group.
    \item \texttt{int MPI\_Comm\_rank(MPI\_COMM\_WORLD, \&rank)}: return the process identifier in the communicator group.
    \item \texttt{int MPI\_Finalize()}: terminates MPI processing, last MPI call.
\end{itemize}

\subsection{Barrier Synchronization}
To syncrhonize among running processes, we must use a barrier:

\texttt{int MPI\_Barrier(MPI\_Comm comm)}: blocks each process in the communicator until all processes have called it.

\subsection{Point-to-Point Communication}

\subsection{Collective Communication}
