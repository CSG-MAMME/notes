\section{OpenMP}

\subsection{OpenMP Basics}
\subsubsection{OpenMP Overview}
OpenMP is an API extension to the C, C++, and Fortran languages to write parallel programs for shared memory machines. It's main components are the following:
\begin{itemize}
    \item \textbf{Constructs:} create threads, share work, and synchronize.
    \item \textbf{Library Routines:} control and query parallel execution environment.
    \item \textbf{Environmnet Variables:} set the execution environment before the program execution starts.
\end{itemize}

\subsubsection{OpenMP Model}
OpenMP uses a \textbf{fork-join} model:
\begin{itemize}
    \item The \textbf{master} thread spawns a team of threads that joins at the end of the parallel region.
    \item Threads in the same team can \textbf{collaborate}.
    \item It uses a relaxed memory model: threads can see different values for the same variable.
    \item Memory consistency is only guaranteed at specific points.
    \item Variables can be shared or private to each thread.
\end{itemize}

\subsubsection{Creating Threads and Accessing Data}

\textbf{The Parallel Construct.}

In order to create threads we use the \texttt{\#pragma omp parallel} construct.
To determine the number of threads to be used for encountered parallel regions we use the \texttt{nthreads-var} ICV.
It is a list, specifying the number of threads per nested parallelism level.
We may use the if construct to add a condition under which we want to spawn (or not) parallelism: \texttt{\#pragma omp parallel if (condition)}.

\textbf{Data-Sharing Attributes.}
Variables can be: \texttt{shared}, \texttt{private}, or \texttt{firstprivate}.
By default, they are \texttt{shared}.
\begin{itemize}
    \item \textbf{\texttt{shared:}} the variable inside the construct is the same than outside. All threads see the same variable, which may need synchronization.
    \item \textbf{\texttt{private:}} if a variable is private, the variable inside the construct is a new variable with undefinded value. Hence, requires no synchronization.
    \item \textbf{\texttt{firstprivate:}} same as private, but the value gets initialized to the original vlaue.
\end{itemize}

\subsubsection{Thread Synchronization}

OpenMP is a shared memory model, hence threads communicate by sharing variables.
Therefore, threads need to synchronize to impose some ordering in their sequence of actions.
\begin{itemize}
    \item \textbf{Thread Barrier:} threads cannot proceed past it until all threads reach the barrier and all work is completed. The \texttt{parallel} construct has an implicit barrier at the end.
    \item \textbf{Critical:} \texttt{\#pragma omp critical} provides a region of mutual exclusion. Only one thread can be working at any given time.
    \item \textbf{Atomic:} \texttt{\#pragma omp atomic} ensures atomic access to a storage location. Usually faster than critical.
    \item \textbf{Reduction:} \texttt{\#pragma omp reduction(+:sum)} all threads accumulate values into a common (shared) variable using one of the following operators \texttt{+,-,*,|,||,\&,\&\&,\^}.
    \item \textbf{Locks:} lastly, the programmer may also use the pre-defined locks in the specification.
\end{itemize}

\subsubsection{Memory Consistency}

As introduced before, OpenMP uses a relaxed memory consistency model.
The \textbf{flush} construct enforces consistency between the temporary view and memory for the variables in the list: \texttt{\#pragma omp flush (list)}.

\subsection{Loop Parallelism in OpenMP}

\subsubsection{The Worksharing Concept}
Worksharing constructs divide the execution of a code region among members of a team.

\subsubsection{Loop Worksharing}
The \texttt{\#pragma omp for} construct spawns a parallel region (when merged as \texttt{\#pragma omp parallel for}) from a for loop.
The iterations are divided among the team, and the default data-sharing attribute is \texttt{shared}.
The \texttt{schedule} clause determines which iterations are executed by each thread:
\begin{itemize}
    \item \texttt{static:} approximately $\#iterations / \#threads$ chunks assigned in round-robin fashion. In the static scheduling, we can also specify a custom chunk size.
    \item \texttt{dynamic:} each thread dynamically grab chunks of $N$ iterations until all have been executed.
    \item \texttt{guided:} variant of \texttt{dynamic}. The size of the chunk decreases as threads grab iterations.
\end{itemize}

The \texttt{nowait} clause can be used to overlap the execution of two (or more) consecutive loops, it removes the implicit barrier at the end of the loop.
To stack two for loops, they need to have the same \textbf{static scheduling} and the same \textbf{iteration space}.

Similarly, the \texttt{collapse} clause allows to distribute work from a set of $n$ nested loops (perfectly nested and rectangular iteration space).

\subsubsection{The Single Construct}

To give work to just one thread, we can use the \texttt{\#pragma omp single} construct.

\subsection{Task Parallelism in OpenMP}

\subsubsection{OpenMP Tasks}
Tasks are work units whose execution may be deferred, threads can cooperate to execute them.
Parallel regions create \textit{implicit} tasks.
One thread that encounters a \texttt{task} construct, packages code and data, and creates a new \textit{explicit} task.
One thread creates them, the rest cooperate to execute them (parallel + single).
The following rules apply:
\begin{itemize}
    \item Global variables are \texttt{shared}.
    \item Variables declared in the scope of the task are \texttt{private}.
    \item The rest are \texttt{firstprivate}.
\end{itemize}

Using the \texttt{if} clause in the task definition, we can generate an \texttt{undeferred} task: if zero the current region is suspended, the task executed, and then resumed again.

Using the \texttt{final} clause, the task and all the generated subtasks, are \texttt{included} tasks, executed sequentially without creating the task context.

Using the \texttt{mergeable} clause, the generated task is an included task and has no task context.

\subsubsection{Task Synchronization}

There are two types of task barriers:
\begin{itemize}
    \item \texttt{taskwait:} suspends the current task, waiting on completion of child tasks of the current task.
    \item \texttt{taskgroup:} suspends the current task waiting on completion of child \textbf{and descendent} tasks.
\end{itemize}
