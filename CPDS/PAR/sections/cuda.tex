\section{Programming with CUDA}

From now onwards, we will refer to the CPU and its memory as the \textbf{host} and to the GPU and it's memory as \textbf{device}.

\subsection{GPU Architecture}
\textbf{CPU vs GPU Architecture:}
\begin{multicols}{2}
    \begin{itemize}
        \item  CPUs are designed for general purpose computing.
        \item Large Caches to reduce impact of long latency.
        \item GPUs are specialized for highly-parallel, compute-intensive computation.
        \item Massive number of threads.
    \end{itemize}
\end{multicols}
The testbed we use in CPDS is the MinoTauro cluster:
\begin{lstlisting}[caption={Output of runing \texttt{nvidia-smi} in the cluster.}]
Thu Dec 19 11:34:34 2019       
+-----------------------------------------------------------------------------+
| NVIDIA-SMI 418.67       Driver Version: 418.67       CUDA Version: 10.1     |
|-------------------------------+----------------------+----------------------+
| GPU  Name        Persistence-M| Bus-Id        Disp.A | Volatile Uncorr. ECC |
| Fan  Temp  Perf  Pwr:Usage/Cap|         Memory-Usage | GPU-Util  Compute M. |
|===============================+======================+======================|
|   0  Tesla K80           Off  | 00000000:04:00.0 Off |                    0 |
| N/A   43C    P0    59W / 149W |      0MiB / 11441MiB |      0%      Default |
+-------------------------------+----------------------+----------------------+
|   1  Tesla K80           Off  | 00000000:05:00.0 Off |                    0 |
| N/A   37C    P0    73W / 149W |      0MiB / 11441MiB |      0%      Default |
+-------------------------------+----------------------+----------------------+
|   2  Tesla K80           Off  | 00000000:85:00.0 Off |                    0 |
| N/A   35C    P0    64W / 149W |      0MiB / 11441MiB |      0%      Default |
+-------------------------------+----------------------+----------------------+
|   3  Tesla K80           Off  | 00000000:86:00.0 Off |                    0 |
| N/A   44C    P0    76W / 149W |      0MiB / 11441MiB |     92%      Default |
+-------------------------------+----------------------+----------------------+
                                                                               
+-----------------------------------------------------------------------------+
| Processes:                                                       GPU Memory |
|  GPU       PID   Type   Process name                             Usage      |
|=============================================================================|
|  No running processes found                                                 |
+-----------------------------------------------------------------------------+
\end{lstlisting}

\textbf{GPU Architecture:}
The GPU structure is divided in different SMs or \textbf{streaming multiprocessors}.
Each SM at the same time has a number of cores:
\begin{itemize}
    \item General purpose load/store units: calculate memory addresses.
    \item Special Function Units (SFU): execute complex instructions such as \textit{sin, cosine, ...}.
\end{itemize}
Separate address spaces between CPU (host memory) and GPU (device memory).

\subsection{CUDA: Devices, Kernel Definitions, and Offloading}
\textbf{Device Management}
An application can query and select the GPUs it wants to use.
Multiple threads may share a single device, and a single thread can manage multiple devices.
A \textbf{kernel} is a function that runs on the device and is called from the host.
\begin{itemize}
    \item Definition using the keyword \texttt{\_\_global\_\_} in the prototype.
    \item Invokation using triple brackets mark a \textbf{kernel launch} (call from host to device code).
    \item The kernel launch is \textbf{asynchronous} (non-blocking).
    \item In order to synchronize we must use \texttt{cudaDeviceSynchronize()}.
\end{itemize}
\begin{lstlisting}[style=cuda,caption={Example of a kernel definition and call.}]
    __global__ void helloworld(void)
    {
    }

    int main(void)
    {
        helloworld<<<1,1>>>();
        cudaDeviceSynchronize();
        return 0;
    }
\end{lstlisting}
\subsection{CUDA: Blocks, Threads, and Indexing}
As introduced before, the triple brackets \texttt{<<<N,M>>>} precede a kernel launch.
In order to make use of the multiple streaming multiprocessors (SM) we replicate the kernels in different blocks:
\begin{itemize}
    \item We use \textbf{N} blocks, each block runs in one SM.
    \item Blocks in a grid are assumed to work independently (no communication among them).
    \item Each block has \textbf{M} threads.
    \item Threads in the same block can synchronize via shared memory.
    \item \textbf{\color{red}{CUDA only allows for synchronization among threads in the same block.}}.
\end{itemize}
\begin{lstlisting}[style=cuda,caption={Example of a kernel definition and call.}]
    // Replication in different SM: N blocks of 1 thread
    helloWorld<<<N, 1>>>();
    // Replication in different threads inside one SM
    helloWorld<<<1, M>>>():
    // Inside the kernel, each instance can be uniquely identified
    int index = threadIdx.x + blockIdx.x * blockDim.x;
\end{lstlisting}
As a summary, functions in CUDA programs can be:
\begin{table}[h!]
    \centering
    \begin{tabular}{p{.4\textwidth} p{.2\textwidth} p{.2\textwidth}}
        \hline
        & \textbf{Executed on the:} & \textbf{Callable from the:} \\[3pt] \hline \hline
        \textbf{\texttt{\color{blue}{\_\_device\_\_} \color{black}float DeviceFunc()}} & device & device \\[3pt]
        \textbf{\texttt{\color{blue}{\_\_global\_\_} \color{black}float KernelFunc()}} & device & host \\[3pt]
        \textbf{\texttt{\color{blue}{\_\_host\_\_} \color{black}float HostFunc()}} & host & host \\[3pt] \hline \hline
    \end{tabular}
\end{table}

In general, thread division in grids, blocks, and threads is specified by the user (with at most three dimensions).
To do so, we make use of the \texttt{dim3} construct:
\begin{lstlisting}[style=cuda,caption={Example of a kernel definition and call.}]
    // Run ceil(n/256) blocks of 256 threads each:
    // Potentially more than one block per SM
    dim3 DimGrid(ceil(n/256), 1, 1);
    dim3 DimBlock(256, 1, 1);
    myKernel<<<DimGrid, DimBlock>>>();
\end{lstlisting}
Threads in a block are divided in 32-thread \textbf{warps} (scheduling units in SM).
At any point in time, \textbf{\textcolor{red}{only one warp per SM is fetching and executing.}}

\subsection{CUDA: Accessing (global) Memory}
There is a different partial view of the device memory between device and host code:
\begin{multicols}{2}
    \begin{itemize}
        \item Device code can read and write \textbf{thread registers}
        \item Device code can read and write \textbf{grid global memory}
        \item Host code can allocate data in \textbf{grid global memory}
        \item Host code can transfer data between \textbf{host} and \textbf{grid}
    \end{itemize}
\end{multicols}
Global variables in grid memory are to be declared using the \texttt{\_\_device\_\_} attribute.
For dynamic memory allocation we use: \texttt{cudaMalloc} and \texttt{cudaFree}.
To transfer data we use \texttt{cudaMemcpy}.
\begin{lstlisting}[style=cuda,caption={Example of a kernel definition and call.}]
int main()
{
    float a[N];
    float result[N];
    float *dev_a, *dev_result;
    cudaMalloc(&dev_a, N * sizeof(float));
    cudaMalloc(&dev_result, N * sizeof(float));
    cudaMemcpy(dev_a, a, N * sizeof(float), cudaMemcpyHostToDevice);
    dim3 DimGrid(ceil(n/256), 1, 1);
    dim3 DimBlock(256, 1, 1);
    myKernel<<<DimGrid, DimBlock>>>(dev_a, dev_result);
    cudaMemcpy(result, dev_result, N * sizeof(float), cudaMemcpyDeviceToHost);
    cudaFree(dev_a);
    cudaFree(dev_result);
}
\end{lstlisting}

\subsection{CUDA: Cooperating Threads and Shared Memory}

As introduced before, threads within the same block may synchronize.
In particular, \texttt{\_\_syncthreads()} forces all warps (all threads in a block) to wait until the rest have reached the same point.
Each thread can:
\begin{itemize}
    \item R/W per-thread registers, per-block shared memory, and per-grid global memory.
    \item Host code can R/W global grid memory.
    \item Threads have shared memory within a block using the \texttt{\_\_shared\_\_} construct.
\end{itemize}
