\section{Gr\"obner Bases}

The contents here contained are a summary of the information provided in class with Chapter 2, sections 1-8 of \textit{Ideals, Varieties, and Algorithms} by Cox \textit{et. al}.

\subsection{Introduction}

\begin{definition}[Polynomial]
    A \textit{polynomial} $f$ in $x_1, \dots, x_n$ with coefficients in a field $k$ is a finite linear combination (with coefficients in $k$) of monomials.
    The set of all polynomials in $x_1, \dots, x_n$ with coefficients in $k$ is denoted $k[x_1, \dots, x_n]$.
\end{definition}

\begin{definition}[Affine Space]
    Given a field $k$ and a positive integer $n$, we define the $n$-dimensional \textit{affine space} over $k$ to be the set
    $$k^n = \{ (a_1, \dots, a_n) | a_1, \dots, a_n \in k \}$$
\end{definition}

\begin{definition}[Affine Variety]
    Let $k$ be a field and let $f_1, \dots, f_s$ be polynomials in $k[x_1, \dots, x_n]$.
    Then the affine variety defined by $f_1, \dots, f_s$ is,
    $$\pmb{V}(f_1, \dots, f_s) = \{ (a_1, \dots, a_n) \in k^n | f_i(a_1, \dots, a_n) = 0 \quad \forall 1 \leq i \leq s \}$$
\end{definition}

\begin{definition}[Ideals]
    A subset $I \subseteq k[x_1, \dots, x_n]$ is an \textit{ideal} if it satisfies:
    \begin{enumerate}
        \item[(i)] $0 \in I$
        \item[(ii)] If $f, g \in I$, then $f + g \in I$
        \item[(iii)] If $f \in I$ and $h \in k[x_1, \dots, x_n]$ then $hf \in I$.
    \end{enumerate}
    Additionally, let $f_1, \dots, f_s$ be polynomials in $k[x_1, \dots, x_n]$.
    Then,
    $$<f_1, \dots, f_s> = \left\{ \sum_{i=1}^s h_i f_i | h_1, \dots, h_s \in k[x_1, \dots, x_n] \right\}$$
    is the \textit{ideal generated} by $f_1, \dots, f_s$.

    Lastly, if $V \subseteq k^n$ is an affine variety, the set
    $$\pmb{I}(V) = \{f \in k[x_1, \dots, x_n] | f(a_1, \dots, a_n) = 0 \quad \forall (a_1, \dots, a_n) \in V\}$$
    is \textit{the ideal of V}.
\end{definition}
