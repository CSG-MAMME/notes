\section{Polytopes, Oriented Matroid, and Gale Duality}

\subsection{Oriented Matroid and Gale Duality Construction}

\begin{definition}[Polyhedron]
    A \textit{polyhedron} in $\mathbb{R}^d$ is any set obtained as the intersection of finitely many halfspaces in $\mathbb{R}^d$.
    Mathematically, it has the form $P = \{x \in \mathbb{R}^d: Ax \leq b \}$ where $A \in \mathbb{R}^{m\times d}$ and $b \in \mathbb{R}^m$.
\end{definition}

\begin{definition}[Polytope]
    A \textit{polytope} is a bounded polyhedron.
    Alternatively, a \textit{polytope} in $\R^d$ is the convex hull of a finite number of points.
    These points are its \textit{vertices}.
\end{definition}

\begin{definition}[Simplices]
    The simplest polytopes are \textit{simplices}.
    The unit $(d-1)$-simplex is defined as $\Delta_{d-1} \coloneqq \text{conv}(\{\pmb{e_1}, \dots, \pmb{e_d}\}) = \{ \pmb{x} \in \R^d : x_1 + \dots + x_d = 1, x_i \geq 0 \}$.
    This family includes a line segment, triangle, and tetrahedron.
\end{definition}

\begin{definition}[Affine Combination, Affine Hull, and Affine Space]
    Affine spaces are closely related to linear vector spaces in the following way:
    \begin{enumerate}
        \item An \textit{affine combination} of vectors $\pmb{p_1}, \dots, \pmb{p_t} \in \R^d$ is a combination of the form $\sum_{i=1}^t \lambda_i \pmb{p_i}$ such that $\sum_{i=1}^t \lambda_i = 1$.
        \item The \textit{affine hull} of $\pmb{p_1}, \dots, \pmb{p_t}$, aff$(\{\pmb{p_1}, \dots, \pmb{p_t}\})$, is the set of all affine combinations of $\pmb{p_1}, \dots, \pmb{p_t}$.
        \item An \textit{affine space} is any set of the form $\lbrace x \in \R^d : Ax = b \rbrace$. It is the translate of the vector space $\lbrace x \in \R^d : Ax = 0 \rbrace$.
    \end{enumerate}
\end{definition}

\begin{definition}[Dimension]
    Using the previous definitions, we may define the dimension of an affine space and a polytope as follows:
    \begin{enumerate}
        \item The \textit{dimension} of an affine space is the dimension of the linear vector space that it is a translate of.
        \item The \textit{dimension} of a polytope $P$ is the dimension of it's affine hull.
    \end{enumerate}
\end{definition}

\begin{definition}[Face of a Polytope]
    The intersection of a polytope $P \subset \R^d$ with a supporting hyperplane of $P$ is called a \textit{face} of $P$.

    The \textit{dimension} of a face is the dimension of the face as a polytope.
    \begin{itemize}
        \item $k$-dimensional faces are called $k$-faces.
        \item $0$-faces are called \textit{vertices}.
        \item $1$-faces are called \textit{edges}.
        \item $(\text{dim}(P) - 1)$-faces are called \textit{facets}.
    \end{itemize}
\end{definition}

\begin{definition}[Simple and Simplicial Polytopes]
    A $d$-dimensional polytope $P$ is \textit{simple} if every vertex of $P$ is incident to $d$ edges of $P$, or equivalently, every vertex of $P$ lies on precisely $d$ facets of $P$.

    A $d$-dimensional polytope $P$ is \textit{simplicial} if for $0 \leq k \leq \text{dim}(P) - 1$, each $k$ face of $P$ is combinatorially isomorphic to a $k$-simplex.
\end{definition}

\subsection{Radon Partition, Gale Diagrams, and Cyclic Polytopes}

The cyclic polytope $C_d(n)$ is a simplical $d$-polytope with $n$ vertices.
It provides an upper bound on face numbers of simplicial polytopes with the same dimension and number of vertices.
\begin{definition}[Cyclic Polytope]
    The $d$-dimensional \textit{cyclic polytope} with $n$ vertices is:
    $$C_d(n) \coloneqq \text{conv}(\{\phi(t_1), \dots,  \phi(t_n)\})$$
    where $\phi : \R \mapsto \R^d, t \mapsto (t, t^2, \dots, t^d)$.
\end{definition}

\begin{obs}
    The facets of $C_d(n)$ satisfy Gale's evenness condition:
    $$ S = \lbrace i_1, \dots, i_d \rbrace \subset [n] \text{ indexes a facet } \Leftrightarrow \forall i,j \not\in S \text{ and } i < j, 2 \text{ divides } |\{k : k \in S, i < k < j\}|$$
\end{obs}

\begin{theorem}[Uppber Bound Theorem]
    Let $P$ be any $d$-dimensional simplicial polytope with $n$ vertices. Then $f_i(P) \leq f_i(C_d(n))$ for every $0 \leq i \leq d-1$.
\end{theorem}

\begin{definition}[k-neighbourness]
    A polytope $P$ is \textit{$k$-neighborly} if any set of $k$ or less vertices of $P$ is the vertex set of a face of $P$.
\end{definition}

\textbf{Gale Transform}

Let $A = (\pmb{a_1}, \dots, \pmb{a_n})$ be a linear matroid of points in $\R^d$.
Now consider, $\bar{A} = \left( \begin{array}{c c c} \pmb{a_1} & \dots & \pmb{a_n} \\ 1 & \dots & 1 \end{array} \right) = (\bar{\pmb{a_1}}, \dots, \bar{\pmb{a_n}})$ vectors in $\R^{d+1}$.
Note that the latter are \textit{affinely independent} iff the former are \textit{linearly independent}.
\begin{definition}[Linear and Affine Value and Dependences]
    Given $A$, and $\bar{A}$ we define the following:
    \begin{itemize}
        \item $\text{LinVal}(\bar{A}) = \{(f(\bar{\pmb{a_1}}), \dots, f(\bar{\pmb{a_n}})) \: | \: f: \R^{d+1} \longrightarrow \R \text{ linear} \}$
        \item $\text{LinDep}(\bar{A}) = \{ \alpha \in \R^{n+1} \: | \: \alpha_1 \pmb{\bar{a_1}} + \dots + \alpha_n \pmb{\bar{a_n}} = 0 \}$
        \item $\text{AffVal}(A) = \{(f(\bar{\pmb{a_1}}), \dots, f(\bar{\pmb{a_n}})) \: | \: f: \R^{d} \longrightarrow \R \text{ affine} \}$
        \item $\text{AffDep}(A) = \{ \alpha \in \R^n \: | \: \alpha_1 \pmb{a_1} + \dots + \alpha_n \pmb{a_n} = 0, \alpha_1 + \dots + \alpha_n = 0 \}$
    \end{itemize}
\end{definition}

\begin{definition}[Gale's Transform and Gale's Diagram]
    Let $B = \{ \pmb{b_1}, \dots, \pmb{b_n} \} \subset \R^{n -d -1}$ be the $n$ ordered vectors such that $\bar{A} B^T = 0$ ($B^T$ is a basis for $\text{ker}(A)$).
    Then $B$ is called a \textit{Gale Transform} of $A$.
    The associated \textit{Gale Diagram} is the vector configuration drawn in $\R^{n -d -1}$.
\end{definition}

\begin{lemma}[Radon's Lemma]
    Let $A$ be a set of $d+2$ points in $\R^d$. Then, there exist two disjoint subsets $A_1, A_2 \subset A$ s.t conv$(A_1) \cap \text{conv}(A_2) \neq \emptyset$.
\end{lemma}

The main goal of the Gale transform is to read the face lattice of a $d$-polytope $P$ from its Gale transfrom.
To do so, we will use the two following statements:

\begin{lemma}
    Let $P = \text{conv}(\{\pmb{a_1}, \dots, \pmb{a_n}\})$. Then $J \subseteq [n]$ is a face of $P$ if and only if
    $$\text{conv}(\{\pmb{a_j} : j \in [n] \backslash J \} \cap \text{aff}(\{\pmb{a_j} : j \in J \}) = \emptyset$$
\end{lemma}
\begin{theorem}
    Let $P = \text{conv}(\{\pmb{a_1}, \dots, \pmb{a_n}\}), \pmb{a_i} \in \R^{d}$, and $B$ its gale transfrom.
    Then,
    $$J \text{ is a face of } P \Leftrightarrow J = [n] \text{ or } \pmb{0} \in \text{relint}(\text{conv}(\{b_k : k \not\in J \}))$$
\end{theorem}

\subsection{Asymptotic Upper Bound Theorem}
