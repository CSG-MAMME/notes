\section{Matroids}

\begin{itemize}
    \item \textbf{Examples:} Vector and point configurations; algebraic matroids; graphical matroids
    \item \textbf{Axiom subsystems:} Independent Sets; bases; circuits; cocircuits; rank functions; flats; hereditarily pure simplical complexes; universally shellable simplical complexes; greedily optimizable independent set systems
    \item \textbf{Operations:} Direct Sum; Deletion; Contraction
    \item \textbf{Oriented Matroids:} Axio systems; circuits; cocircuits
\end{itemize}

\subsection{Axiom Subsystems:}

\begin{definition}[Indepndent Sets]
    Let $E$ be a finite set called the \textit{ground set}.
    Then $\I \subset 2^E$ forms the \textit{independent sets} of a \textit{matroid} $M$ on $E$ - $M = (E, \I)$ - if $\I$ satisfies:
    \begin{enumerate}
        \item[I1.] $\emptyset \in \I$
        \item[I2.] $I_1 \in \I$ and $I_2 \subset I_1$ implies $I_2 \in \I$
        \item[I3.] (\textit{Exchange Axiom}) $I_1, I_2 \in \I$ and $|I_2| > |I_1|$ implies there exists $e \in I_2 - I_1$ with $I_1 \cup \lbrace e \rbrace \in \I$.
    \end{enumerate}
\end{definition}    

\begin{definition}[Bases]
    The \textit{bases} of a \textit{matroid} $M$ on $E$, $\B(M)$, are the set of inclusion-maximal independent sets (adding another element makes them dependent, \textit{i.e.} not in $\I$).
    They also satisfy two axioms:
    \begin{enumerate}
        \item[B1.] $\B \neq \emptyset$
        \item[B2.] (\textit{Exchange Axiom}) Given $B_1, B_2 \in \B$ and $x \in B_1 - B_2$, there exists $y \in B_2 - B_1$ with
            $$ (B1 - \{ x\} \cup \{ y\} \in \B$$
    \end{enumerate}
\end{definition}

\begin{definition}[Circuits]
    The \textit{circuits} $\C$ of a matroid are the inclusion-minimal \textit{dependent} sets (\textit{i.e.} all proper subsets are independent).
    They satisfy the following axioms:
    \begin{enumerate}
        \item[C1.] $\emptyset \not\in \C$
        \item[C2.] $C_1, C_2 \in \C$ and $\C_1 \subset \C_2$ implies $C_1 = C_2$.
        \item[C3.] (\textit{Circuit Elimination}) $C_1, C_2 \in \C$ with $C_1 \neq C_2$ and $e \in C_1 \cap C_2$ implies there exists $C_3 \in \C$ with $C_3 \subset (C_1 \cup C_2) - \{ e \}$.
    \end{enumerate}
\end{definition}

\begin{definition}[Rank Function]
    Given a matroid $M$ on $E$, for $A \subset E$ we define the \textit{rank function}
    $$ r(A) \coloneqq \max \left\lbrace |I|: I \in \I(M) \text{ and } I \subseteq A \right\rbrace.$$
    It satisfies the rank axioms:
    \begin{enumerate}
        \item[R1.] $0 \leq r(A) \leq |A|$
        \item[R2.] $A_1 \subset A_2$ implies $r(A_1) \leq r(A_2)$
        \item[R3.] (\textit{Semimodularity}) $r(A \cup B) + r(A \cap B) \leq r(A) + r(B)$
    \end{enumerate}
    It is useful to recall that $\I = \{ I \subset E : r(I) = |I| \}$.
\end{definition}

\begin{definition}[Corank]
    Given a matroid $M$ on $E$, $A \subset E$, and a rank function $r$, $corank(A) = r(E) - r(A)$.
\end{definition}

\begin{definition}[Closure and Flats]
    The \textit{closure} operation
    $$\bar{A} \coloneqq \{ e \in E : r(A \cup \{ e \}) = r(A) \}$$
    We say a set $A \subset E$ is \textit{closed} if $A = \bar{A}$.
    Closed sets are also colled \textit{flats}.
    The closure operation satisfies the following axioms:
    \begin{enumerate}
        \item[CL1.] $A \subset \bar{A}$
        \item[CL2.] $\bar{\bar{A}} = \bar{A}$
        \item[CL3.] $A_1 \subseteq A_2$ implies $\bar{A_1} \subseteq \bar{A_2}$.
        \item[CL4.] (\textit{Exchange}) If $x,y \in E$ and $A \subset E$ have $y \in \bar{A \cup \{x \}} - \bar{A}$ then $x \in \bar{A \cup \{ y \}}$.
    \end{enumerate}
\end{definition}

\begin{definition}[Geometric Lattice of Flats]
    The poset $L(M)$ of all flats of $M$, ordered by inclusion, is a \textit{geometric lattice} such that:
    \begin{itemize}
        \item Has meets, $x \wedge y$, and joins $x \vee y$.
        \item It is \textit{(upper-)semimodular}, ranked, and satisfying
            $$ r(x \vee y) + r(x \wedge y) \leq r(x) + r(y) $$
        \item It is \textit{atomic}: every element is the join of the \textit{atoms}(cover, immediately greater element, of a minimal element) below it.
    \end{itemize}
\end{definition}

\begin{definition}[Loops and Coloops]
    The \textit{loops} of a matroid $M$ are the set of elements that belong to no independent set.

    The \textit{coloops} of a matroid $M$ are the set of elements that belong to e\textit{every} basis (can be added to any independent set, keeping it independent).
\end{definition}

\begin{definition}[Simple Matroids]
    A matroid is \textit{simple} if it has no:
    \begin{itemize}
        \item \textit{loops}: elements lying in no independent set.
        \item \textit{parallel} elements: elements $e, e'$ with $e' \in \bar{\{ e \}}$.
    \end{itemize}
\end{definition}

\begin{definition}[Simplical Complexes]
    A set fulfulinig $I1$ and $I2$ is said to be an \textit{abstract simplical complex} on $E$.
    We can replace $I3$ for various others:
    \begin{enumerate}
        \item[I3'] \textit{Pure Simplical Complexes:} a simplical complex is \textit{pure} (of dimension $r-1$) if every maximal face has the same cardinality $r$.
            Then, it must hold for every subset $A \subset E$,
            $$I|_A \coloneqq \{ I \in \I : I \subset A \}$$
            is a \textit{pure} simplical complex.
        \item[I3''] \textit{Shellable:} A simplical complex is \textit{shellable} if there exists an ordering $F_1, \dots, F_t$ of its \textit{facets}(maximal faces) having the following property: for each $j \geq 2$, the facet $F_j$ intersects the subcomplex generated by the previous ones in  a pure subcomplex of dimension 1.
            Then it must hold:

            \textit{Every ordering of the facets is a shelling ordering}.
    \end{enumerate}
\end{definition}

\subsection{Motivating Examples:}

\begin{definition}[Linear Matroid]
    Let $V$ be a vector space, then a linear matroid in $V$ is defined by
    \begin{equation*}
        \begin{split}
            E & = \lbrace \text{ finite subset of } V \rbrace \\
            \I & = \lbrace \text{ linearly independent subsets of } E \rbrace
        \end{split}
    \end{equation*}
\end{definition}

\begin{definition}[Graphical Matroid]
    Let $G = (V, E)$ be a graph,
    \begin{equation*}
        \begin{split}
            E & = \lbrace \text{ edges of } G \rbrace = E \\
            \I & = \lbrace \text{ edges of forests of } \mathcal{G} \rbrace
        \end{split}
    \end{equation*}
\end{definition}

\begin{definition}[Transversal Matroid]
    Let $G = (U, V, E)$ be a bipartite graph,
    \begin{equation*}
        \begin{split}
            E & = \lbrace \text{ bottom vertices of } G \rbrace = U \\
            \I & = \lbrace \text{ endpoints of maximal matchings } \mathcal{V} \rbrace
        \end{split}
    \end{equation*}
\end{definition}

\subsection{Oriented Matroids}

To obtain a \textit{signed} matroid from a regular one, we just have naturally add a sign.
For instance, in a graphical matroid, the oriented matroid would correspond to the directed graph.

For the signed circuits, we proceed similarly.
In the graphical case, a plus or a minus would mean that the edge in the loop is traversed in its natural direction or the opposite one.
In a linear one, the sign of the circuit would correspond to the coeficients of minimal linear dependences.

\begin{definition}[Signed Covectors]
    Signed covectors are evaluations of normal vectors.
    \textit{I.e.} the sign relative to its peers induced by the hyperplane of the normal vector.
\end{definition}

\subsection{Operations on Matroids}

\begin{definition}[Direct Sum]
    Given matroids $M_1, M_2$ on the ground sets $E_1, E_2$, their \textit{direct sum} $M_1 \oplus M_2$ is the matroid on ground set $E_1 \sqcup E_2$ having independent sets
    $$\I(M_1 \oplus M_2) = \I(M_1) \times \I(M_2)$$
\end{definition}

\begin{definition}[Deletion]
    Given a matroid $M$ with ground set $E$ and $e \in E$ is not a coloop, the \textit{deletion} $M \backslash e$ is the matroid on $E - \{ e \}$ with independent sets
    $$\I(M \backslash e) = \lbrace I \in \I(M) : e \not\in I \rbrace$$
    We can express the \textit{restriction} in terms of \textit{deletion},
    $$M|_A \coloneqq M \backslash (E - A)$$
\end{definition}

\begin{definition}[Contraction]
    Given a matroid $M$ on $E$, and $e \in E$ not a loop, the \textit{contraction} $M / e$ is the matroid on ground set $E - \{ e \}$ with independent sets,
    $$\I(M / e) = \{ I - \{ e\}: e \in I \in \I(M) \}$$
\end{definition}

\begin{definition}[Dual Matroid]
    Given a matroid $M$ on ground set $E$, its \textit{dual} matroid $M^*$ is defined by its set of bases,
    $$\B(M^*) = \{ E - B : B \in \B(M) \}$$
    Some interesting properties of the dual matroid:
    \begin{itemize}
        \item $$(M \backslash e )^* = M^* / e^*$$
        \item Loops are dual to coloops.
        \item Circuits (minimal linear dependences) are dual to cocricuits (minimal value vectors with minimal support).
    \end{itemize}
\end{definition}
